\documentclass[11pt]{article}
\pagestyle{empty}
\usepackage[a4paper, margin=0.75in]{geometry}
\usepackage{enumitem}
\renewcommand{\familydefault}{\sfdefault}

\usepackage[explicit]{titlesec}
\titlespacing*{\section}
{0pt}{0.7\baselineskip}{0pt}
\titlespacing*{\subsection}
{0pt}{0pt}{0pt}
%\titleformat*{\section}{\large\bfseries\sffamily}
%\titleformat*{\subsection}{\normalsize\bfseries\sffamily}
\titleformat{\section}
  {\bfseries}{\thesection}{1em}{\MakeUppercase{#1}}
 
 \titleformat{\subsection}
  {\bfseries}{\thesection}{1em}{#1}
 
\begin{document}
\setlist[itemize]{noitemsep, topsep=0pt, leftmargin=*}
\setlength{\parindent}{0pt}


\vspace*{5\baselineskip}

\section*{Dual Degree Project}

\subsection*{Implementation of cognitive radio on the USRP kit $\vert$  June 2013 - till date}
\emph{Guide: Prof S N Merchant, Dept. of Electrical Engineering, IIT Bombay} 
\begin{itemize}
\item Carried out energy detection spectrum sensing to find the lowest energy frequency band  using Python
\item Working on a cognitive OpenBTS system 
\end{itemize}


\section*{Programming skills}


\textbf{Languages}: C, Matlab, Python, bash scripting, Assembly for the Intel 8085 microprocessor, Octave, Verilog, Ruby, SQL, XML  \\
\textbf{Operating systems}: Unix, Linux and Mac OS X


\section*{Course projects}

\subsection*{Scalable video coding using wavelets $\vert$  Feb-Apr 2013}
\emph{Guide: Prof V M Gadre, Dept. of Electrical Engineering, IIT Bombay} 
\begin{itemize} 
\item Compressed three different spatial resolutions of a video together into a single bitstream  
\item At the receiving end, uncompressed the best resolution for the bit rate available 
\end{itemize}

\subsection*{Principal Component Analysis in face recognition $\vert$  Oct-Nov 2012} 
\emph{Guide: Prof V Rajbabu, Dept. of Electrical Engineering, IIT Bombay}  
\begin{itemize}
\item Implemented an iterative algorithm of using PCA in face recognition  
\end{itemize}

\subsection*{Design and test an algorithm for restoring a brain image $\vert$  Sep-Nov 2012}
\emph{Guide: Prof Arjun Arunachalam, Dept. of Electrical Engineering, IIT Bombay}  
\begin{itemize}
\item Implemented an algorithm to remove noise artifacts from a brain image  
\end{itemize}

%\subsection*{Mini UID for IIT Bombay Campus $\vert$  Oct-Nov 2009}
%\emph{Guide: Prof Deepak Phatak, Dept. of Computer Science and Engineering, IIT Bombay} 
%\begin{itemize}
%\item Automated fingerprint matching for the purposes of registration, verification and attendance 
%\end{itemize}



\section*{Key course assignments}

\subsection*{Course: Advanced computing for electrical engineers $\vert$ Sep 2012}
\emph{Guide: Prof Virendra Singh, Dept. of Electrical Engineering, IIT Bombay}
\begin{itemize}
\item Implemented stack, queue, double ended queue, linked list, doubly linked list, self-adjusting lists
\item Implemented 2-3 tree, splay tree,  huffman tree and AVL tree
\item Used C programming language for the implementation
\end{itemize}


\section*{Seminars}

\subsection*{Measurement of interference temperature $\vert$  Jan-Apr 2013} 
\emph{Guide: Prof S N Merchant, Dept. of Electrical Engineering, IIT Bombay}
\begin{itemize}  
\item Surveyed various ways of measuring interference temperature efficiently 
\end{itemize}
  
\subsection*{LED's for high speed applications (over 100 Mbps) $\vert$  Mar-Apr 2013}
\emph{Guide: Prof Joseph John, Dept. of Electrical Engineering, IIT Bombay}  
\begin{itemize}  
\item Presented a seminar on how LED's could be used for high speed fiber optic communications. LED's are cheaper, rugged and safer to handle compared to laser diodes 
\end{itemize}



\section*{Extra curricular}

\begin{itemize}
\item Won a silver medal in the All India Computer Knowledge Competition 2006 
\item `9/10' in the course EE 717: Advanced Computing for Electrical Engineers 
\item \textbf{Organizer, Infrastructure Team},Techfest 2010 
\item \textbf{Social Service}: surveyed water and electrical resources of remote villages in Maharashtra 
\item \textbf{Interests}:  Computer programming for technical problems, communications, problem solving, functional programming 
\end{itemize}


\end{document}