\documentclass[12pt]{article}
\pagestyle{empty}
\usepackage[a4paper, margin=0.75in]{geometry}
\usepackage{enumitem}
\renewcommand{\familydefault}{\sfdefault}

\usepackage[explicit]{titlesec}
\titlespacing*{\section}
{0pt}{1.4\baselineskip}{0.7\baselineskip}
\titlespacing*{\subsection}
{0pt}{0.7\baselineskip}{0pt}
%\titleformat*{\section}{\large\bfseries\sffamily}
%\titleformat*{\subsection}{\normalsize\bfseries\sffamily}
\titleformat{\section}
  {\bfseries}{\thesection}{1em}{\MakeUppercase{#1}}
 
 \titleformat{\subsection}
  {\bfseries}{\thesection}{1em}{#1}
 
\begin{document}
\setlist[itemize]{noitemsep, topsep=3pt, leftmargin=*}
\setlength{\parindent}{0pt}
 


\vspace*{5\baselineskip}

\section*{Dual Degree Project}

\subsection*{Implementation of cognitive radio on the USRP kit $\vert$  June 2013 - till date}
\emph{Guide: Prof S N Merchant, Dept. of Electrical Engineering, IIT Bombay} 
\begin{itemize}
\item Carried out energy detection spectrum sensing to find the lowest energy frequency band using Python
\item Set up calls and messages on a software defined GSM network named OpenBTS
\item Carried out a field testing of an OpenBTS network to check the interference with other nearby networks 
\item Working on a cognitive OpenBTS system 
\end{itemize}

\section*{Programming skills}
\textbf{General purpose}: C, Python, Assembly for the Intel 8085 microprocessor, Verilog, Ruby \\
\textbf{Numerical computing}: Matlab, Octave, SciPy \\
\textbf{Miscellaneous}: bash scripting, SQL, XML, Javascript, LaTeX \\
\textbf{Operating systems}: Unix, Linux and Mac OS X

\section*{Course projects}
\subsection*{Scalable video coding using wavelets $\vert$  Feb-Apr 2013}
\emph{Guide: Prof V M Gadre, Dept. of Electrical Engineering, IIT Bombay} 
\begin{itemize} 
\item Compressed three different spatial resolutions of a video together into a single bitstream using Matlab
\item At the receiving end, uncompressed the best resolution for the bit rate available 
\end{itemize}

\subsection*{Principal Component Analysis in face recognition $\vert$  Oct-Nov 2012} 
\emph{Guide: Prof V Rajbabu, Dept. of Electrical Engineering, IIT Bombay}  
\begin{itemize}
\item Implemented an iterative algorithm of using PCA in face recognition using Matlab
\end{itemize}

\subsection*{Design and test an algorithm for restoring a brain image $\vert$  Sep-Nov 2012}
\emph{Guide: Prof Arjun Arunachalam, Dept. of Electrical Engineering, IIT Bombay}  
\begin{itemize}
\item Implemented an algorithm to remove noise artifacts from a brain image using Matlab
\item Used the non-linear conjugate gradient method to optimize the estimate
\end{itemize}

\subsection*{A simple AM voice transmitter $\vert$ Aug-Oct 2011} 
\emph{Guide: Prof S N Merchant, Dept. of Electrical Engineering, IIT Bombay}
\begin{itemize} 
\item Developed an AM voice transmitter with a carrier frequency of 1 MHz, taking input from a music player via a 3.5 mm jack
\end{itemize}

%\subsection*{One way selective data transmission $\vert$ Mar-Apr 2011}
%\emph{Guide: Prof Udayan Ganguly, Dept. of Electrical Engineering, IIT Bombay}
%\begin{itemize}
%\item Designed a Verilog based system to send data to multiple locations selectively and implemented it on a CPLD board
%\end{itemize}

\subsection*{Mini UID for IIT Bombay Campus $\vert$  Oct-Nov 2009}
\emph{Guide: Prof Deepak Phatak, Dept. of Computer Science and Engineering, IIT Bombay} 
\begin{itemize}
\item Automated fingerprint matching for the purposes of registration, verification and attendance
\end{itemize}

\section*{Key course assignments}
\subsection*{Course: Advanced computing for electrical engineers $\vert$ Sep 2012}
\emph{Guide: Prof Virendra Singh, Dept. of Electrical Engineering, IIT Bombay}
\begin{itemize}
\item Implemented stack, queue, double ended queue, linked list, doubly linked list, self-adjusting lists
\item Implemented 2-3 tree, splay tree,  huffman tree and AVL tree
\item Used C programming language for the implementation
\end{itemize}

\subsection*{Course: Speech Processing $\vert$ Autumn 2012}
\emph{Guide: Prof Preeti Rao, Dept. of Electrical Engineering, IIT Bombay}
\begin{itemize}
\item Synthesized speech signals using Matlab and used DTFT to analyze them
\item Analyzed speech signals using Linear Prediction and also re-synthesized them.
\item Estimated the pitch of a speech signal using Cepstrum estimation
\item Used Praat software to extract speech signals from .wav files
\end{itemize}


\section*{Seminars}
\subsection*{Measurement of interference temperature $\vert$  Jan-Apr 2013} 
\emph{Guide: Prof S N Merchant, Dept. of Electrical Engineering, IIT Bombay}
\begin{itemize}  
\item Surveyed various ways of measuring interference temperature efficiently 
\item Became familiar with the concept of Cognitive Radio
\end{itemize}
  
\subsection*{LED's for high speed applications (over 100 Mbps) $\vert$  Mar-Apr 2013}
\emph{Guide: Prof Joseph John, Dept. of Electrical Engineering, IIT Bombay}  
\begin{itemize}  
\item Presented a seminar on how LED's could be used for high speed fiber optic communications. LED's are cheaper, rugged and safer to handle compared to laser diodes 
\end{itemize}

\section*{Relevant courses}

\subsection*{Communications}
Digital Communications, Fibre Optic Communications, Communication Systems, Probability and Random Processes, Radiating Systems

\subsection*{Computing and math}
Advanced Computing for Electrical Engineers, Microprocessors, Microprocessors Lab, Optimization Models, Optimization Techniques

%\subsection*{Signal Processing}
%Advanced Topics in Signal Processing, Wavelets, Speech Processing, Image Processing,
%Digital Signal Processing

\section*{Extra curricular}
\begin{itemize}
\item Won a silver medal in the All India Computer Knowledge Competition 2006 
\item `9/10' in the course EE 717: Advanced Computing for Electrical Engineers 
\item \textbf{Organizer, Infrastructure Team}, Techfest 2010 
\item \textbf{Social Service}: surveyed water and electrical resources of remote villages in Maharashtra 
\item \textbf{Social Service}: taught math and physics to 6th standard students 
\item \textbf{Interests}:  Computer programming for technical problems, communications, wireless applications, problem solving, functional programming 
\end{itemize}


\end{document}